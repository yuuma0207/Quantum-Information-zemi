%ctrl+alt+m:open Math Preview
\documentclass[a4paper,11pt,uplatex]{jsarticle}%titlepage
%:/usr/local/texlive/texmf-local/tex/latex/report/report.sty
\usepackage{myreport}
\title{量子力学入門 第6章前半}
\author{松本侑真}
\date{\today}
\begin{document}
\maketitle
\tableofcontents
\newpage
\setcounter{section}{5}
\section{量子エンタングルメント}
\subsection{はじめに}
1935年、Einstein, Podolsky, Rosenは量子力学の記述が不完全である証拠として、ある思考実験を提出した。
この思考実験はEPRパラドックスと呼ばれ、やがてその本質は量子力学における量子エンタングルメント(量子もつれ)という
非局所的な性質にあることが理解されるようになった。量子力学的にエンタングルした2つの系を考えると、それらはどれだけ遠く離れていたとしても、
もはや独立には振舞わない(局所性の仮定が成立しない)のである。
量子エンタングルメントは元から量子力学の根幹にかかわる基礎科学的な問題として議論されていたが、1982年に量子エンタングルメントは量子力学的な系が持つ
特徴であることが実験的に確認されるとパラダイムシフトが起こった。
量子エンタングルメントを単に基礎科学的な問題としてだけではなく、それをいかにして利用するかといった研究が行われ始めた。

1980年代に入り、量子アルゴリズムや量子暗号プロトコルなどが相次いで提案されると、量子エンタングルメントは量子情報処理で利用可能な重要な資源の1つであると認識されるようになった。
1990年代に入ると、量子エンタングルメントの研究がきわめて活発に行われるようになる。量子情報処理が対象とする系が持っている量子エンタングルメントという性質を明らかにすることが、
量子情報処理技術の発展のために必要不可欠だからである。それと同時に、量子エンタングルメントの新しい性質を発見することが新しい量子情報処理技術の発見に繋がるとの期待もあった。
こうして、量子エンタングルメント理論は量子情報理論とともに急速に進展した。

特に、\textbf{局所操作と古典通信}(local operations and classical communication: \textbf{LOCC})の考え方を導入することにより、
量子エンタングルメントの定量化が可能となり、量子エンタングルメントに関する理解は大きく進展した。
AとBの2者を考え、その間で量子ビットのやり取り(量子通信)はできないものの、通常の情報のやり取り(古典通信)はできるものとする。
これがLOCCである。LOCCはエンタングルした状態を生成できないという性質を持っている。
したがって、もしLOCCだけでは不可能だった情報処理がLOCCとエンタングル状態を組み合わせることで可能になったとすれば、それはエンタングル状態が持つ威力の発現だということができる。
LOCCのように制限された操作を考えることで、量子エンタングルメントの持つ威力が浮き彫りになるのである。またLOCCは、それそのものが量子情報プロトコルとしての有用性を持つ。
実際、量子テレポーテーションをはじめ、量子エンタングルメントを利用した量子情報処理プロトコルの多くがLOCCである。

3者以上の多者からなる量子系では、さまざまな種類の量子エンタングルメントが存在し、多彩な現象を見せるようになる。
また、現実の系では、量子状態はデコヒーレンスの影響により混合状態になってしまうので、混合状態における量子エンタングルメントの性質を理解しておくことは、
現実の量子情報処理技術にとって重要なことである。

\subsection{エンタングルメントの基礎}
\subsubsection{量子相関と古典相関}
AとBの2つの量子ビットを考える。どちらの量子ビットも$\ket{0}$である状態は$\ket{0}_{\text{A}}\ket{0}_{\text{B}}$、
どちらも$\ket{1}$である状態は$\ket{1}_{\text{A}}\ket{1}_{\text{B}}$である。この2つの状態を重ね合わせた状態
\begin{equation}
  \ket{\Psi} = \frac{\ket{0}_{\text{A}}\ket{0}_{\text{B}} + \ket{1}_{\text{A}}\ket{1}_{\text{B}}}{\sqrt{2}}
\end{equation}
はエンタングルした状態(もつれた状態)と呼ばれる。

いま、AとBそれぞれの量子ビットに対し、$\ket{0}$なのか$\ket{1}$なのかを決める測定を行ったとする。もしAの測定結果が$\ket{0}$だとすると、$\ket{\Psi}$は$\ket{0}_{\text{A}}\ket{0}_{\text{B}}$に収縮するので、
Bの測定結果も必ず$\ket{0}$になる。同様に、Aの測定結果が$\ket{1}$だとすると、Bの測定結果も必ず$\ket{1}$になる。
AとBの距離が例えどれだけ離れていようとも、AとBの測定結果は常に同じになる。
しかし、これ自体はあまり驚くべき現象ではない。例えば、いまAとBの2つの箱があり、2つの箱にはそれぞれ同じ数字を書いた紙が初めから入っているとする。
ただし、0と1のどちらの数字が書いてあるかは知らないとする。このとき、Aの箱を開けて0と書いた紙を見つければ、Bの箱を開けたときも0と書いてある紙を見つけるはずである。
このようなありふれた場合でも、距離に関係なくAとBの測定結果は常に同じになる。

ところが、箱の中の紙とは異なり、エンタングルした状態$\ket{\Psi}$は遥かに強い相関を示すのである。$\ket{0}$と$\ket{1}$を$\theta$だけ回転させた直交基底を
\begin{equation}
  \begin{cases}
    \ket{\theta}         & = \;\;\;\cos\theta\ket{0} + \sin\theta\ket{1} \\
    \ket{\theta_{\perp}} & = -\sin\theta\ket{0} + \cos\theta\ket{1}
  \end{cases}
\end{equation}
とすると$\ket{\Psi}$は
\begin{equation}
  \ket{\Psi} = \frac{\ket{0}_{\text{A}}\ket{0}_{\text{B}} + \ket{1}_{\text{A}}\ket{1}_{\text{B}}}{\sqrt{2}} 
             = \frac{\ket{\theta}_{\text{A}}\ket{\theta}_{\text{B}} + \ket{\theta_{\perp}}_{\text{A}}\ket{\theta_{\perp}}_{\text{B}}}{\sqrt{2}}
             \label{eq:entangled_unitary}
\end{equation}
と書き換えることができる。したがって、AとBそれぞれに、$\ket{\theta}$なのか$\ket{\theta_{\perp}}$なのかを決める測定を行った場合でも、
両者の測定結果は常に一致するのである。このように、エンタングルした状態に対する測定結果は、基底の取り方に関係なく一致するという強い相関を示し、
これは\textbf{量子相関}(quantum correlation)と呼ばれる。これに対し、箱の中の紙の例は\textbf{古典相関}(classical correlation)と呼ばれる。
前者のエンタングルした状態$\ket{\Psi}$の密度行列表示は
\begin{equation}
  \ketbra{\Psi} = \frac{1}{2}\qty(\ketbra{00} + \ketbra{00}{11} + \ketbra{00}{11} + \ketbra{11})
\end{equation}
であるが、後者の古典相関を持つ系の密度行列表示は
\begin{equation}
  \sigma = \frac{1}{2}\qty(\ketbra{00} + \ketbra{11})
\end{equation}
の混合状態であり、$\ket{\Psi}$の状態とは明確に区別される。

\subsubsection{積状態と最大エンタングル状態}

もっと一般の、AとBの2つの$d$準位系における純粋状態$\ket{\psi}$を考える。
Schmidt分解より、$\ket{\psi}$はAとBの適当な局所ユニタリ変換($U\otimes V$)を使って
\begin{equation}
  (U\otimes V)\ket{\psi} = \sum_{i=0}^{d-1}\sqrt{p_i}\ket{i}_{\text{A}}\ket{i}_{\text{B}}
\end{equation}
と書くことができる。Schmidt係数は大きい順に並んでいるとする:
\begin{equation}
  p_0 \geq p_1 \geq \cdots \geq p_{d-1}\;。 
\end{equation}
$p_0=1$のとき、$p_1=p_2=\cdots=p_{d-1}=0$であり、
\begin{equation}
  \ket{\psi} = \qty(U^{\dagger}\otimes V^{\dagger})\qty(\ket{0}_{\text{A}}\otimes\ket{0}_{\text{B}}) = \ket{f}_{\text{A}}\otimes\ket{g}_{\text{B}}
\end{equation}
になるが、このようにAの状態とBの状態の積でかける状態を\textbf{積状態}(product state)と呼ぶ。
積状態の縮約密度演算子は純粋状態である:
\begin{equation}
  \sigma_{A} = \ketbra{f},\,\sigma_{B} = \ketbra{g}\;。
\end{equation}
一方、$p_0 < 1$のとき、$\ket{\psi}$の右辺は2項以上の和になり、Aの状態とBの状態の積として書くことができない。
\footnote{$i\neq j$のとき、$\ket{i}\neq \ket{j}$であるため必要十分条件である。}
この状態がエンタングル状態(entangled state)である。特に、すべてのSchmidt係数が等しく$p_i=1/d$のときの状態
\begin{equation}
  \ket{\Psi} = \frac{1}{\sqrt{d}}\sum_{i=0}^{d-1}\ket{i}_{\text{A}}\ket{i}_{\text{B}}
\end{equation}
を最大エンタングル状態(maximally entangled state)と呼ぶ。最大エンタングル状態の縮約密度演算子は
\begin{equation}
  \sigma_A = \sigma_B = \frac{1}{d}I
\end{equation}
であり、規格化定数を除いて単位演算子(すなわち完全混合状態)となる。

$X$をAに作用する任意の演算子とし、それを
\begin{equation}
  X = \sum_{k,l=0}^{d-1}\ket{k}X_{kl}\bra{l}
\end{equation}
で表す。このとき、最大エンタングル状態$\ket{\Psi}$に関して、
\begin{align}
  (X\otimes I)\ket{\Psi} &= \frac{1}{\sqrt{d}}\qty(\sum_{k,l}\ket{k}X_{kl}\bra{l}\otimes I)\sum_{i}\ket{i}_{\text{A}}\ket{i}_{\text{B}} 
                         = \frac{1}{\sqrt{d}}\sum_{k,i} X_{ki}\ket{k}_{\text{A}}\ket{i}_{\text{B}} \\
                         &= \frac{1}{\sqrt{d}}\qty(I\otimes\sum_{i,l} \ket{i}X_{il}\bra{l})\sum_{k}\ket{k}_{\text{A}}\ket{k}_{\text{B}}
                         = \qty(I\otimes X^\top)\ket{\Psi} 
\end{align}
が成立する。
これより、$X$を実直交行列$O$とすれば、
\begin{equation}
  (O\otimes O)\ket{\Psi} = \qty(O\otimes O^\top)\ket{\Psi} = \ket{\Psi}
\end{equation}
となり、式\eqref{eq:entangled_unitary}が確かめられる。
\footnote{実直交ではないユニタリ行列で変換したら成り立たないの?本では、基底の取り方に関係なく一致するとあったが、あくまで実直交行列による変換のみ?}
また、Schmidt分解の基底の取り方の任意性を考慮すると、最大エンタングル状態は任意の局所ユニタリ変換によって変換される。すなわち、
$d$準位系のすべての最大エンタングル状態は$(U \otimes V)\ket{\Psi}$の形をしていることになるが、これは
\begin{equation}
  (U \otimes V)\ket{\Psi} = (U\otimes I)(I\otimes V)\ket{\Psi} = (UV^{\top}\otimes I)\ket{\Psi} = (I\otimes VU^{\top})\ket{\Psi}
\end{equation}
と書き換えられる。
$UV^{\top}$や$VU^{\top}$は局所ユニタリ変換なので、すべての最大エンタングル状態は、AまたはBだけの局所ユニタリ変換で相互変換可能であることになる。
以下の4つの状態は、いずれも2つの量子ビットにおける最大エンタングル状態である:
\begin{equation}
  \ket{\phi^{\pm}}_{\text{AB}} = \frac{\ket{0}_{\text{A}}\ket{0}_{\text{B}} \pm \ket{1}_{\text{A}}\ket{1}_{\text{B}}}{\sqrt{2}},\quad
  \ket{\psi^{\pm}}_{\text{AB}} = \frac{\ket{0}_{\text{A}}\ket{1}_{\text{B}} \pm \ket{1}_{\text{A}}\ket{0}_{\text{B}}}{\sqrt{2}}\;。
\end{equation}
これらは\textbf{Bell基底}(Bell basis)と呼ばれ、2つの量子ビットにおけるCONS(完全規格直交系)を構成している。Bell状態は最大エンタングル状態なので、
A側にPauli行列のユニタリ変換を施すことで、以下のように相互変換ができる:
\begin{align}
  \ket{\phi^{+}}_{\text{AB}} &= (I\otimes I)\ket{\phi^{+}}_{\text{AB}}\;, \\
  \ket{\psi^{+}}_{\text{AB}} &= (\sigma_x\otimes I)\ket{\phi^{+}}_{\text{AB}}\;, \\
  \ket{\phi^{-}}_{\text{AB}} &= (i\sigma_y\otimes I)\ket{\phi^{+}}_{\text{AB}}\;, \\
  \ket{\psi^{-}}_{\text{AB}} &= (\sigma_z\otimes I)\ket{\phi^{+}}_{\text{AB}}\;。
\end{align}

\newpage

\subsubsection{量子テレポーテーション}
エンタングルメントを利用したプロトコルの中で、最も基本的で重要なものが\textbf{量子テレポーテーション}(quantum teleportation)である。
量子テレポーテーションは、エンタングル状態と古典通信を用いて量子状態を遠隔地へと転送するプロトコルである。いま、送信者と受信者はエンタングル状態
\begin{equation}
  \ket{\phi^{+}}_{\text{AB}} = \frac{\ket{0}_{\text{A}}\ket{0}_{\text{B}} + \ket{1}_{\text{A}}\ket{1}_{\text{B}}}{\sqrt{2}}
\end{equation}
を保持しているとする。また、送信者が持っている量子ビットXの状態$\ket{\psi}_{\text{X}} = a\ket{0}_{\text{X}} + b\ket{1}_{\text{X}}$を
受信者へ転送したいとする。このとき、X、A、Bの3つの量子ビット全体の状態を、4つのBell基底を使って書き換えると、
\begin{align}
  (a&\ket{0}_{\text{X}}+b\ket{1}_{\text{X}})\otimes  \frac{\ket{0}_{\text{A}}\ket{0}_{\text{B}} + \ket{1}_{\text{A}}\ket{1}_{\text{B}}}{\sqrt{2}} \notag\\
  =& \frac{1}{2}\left[\ket{\phi^{+}}_{\text{XA}}\otimes(a\ket{0}_{\text{B}}+b\ket{1}_{\text{B}}) + \ket{\phi^{-}}_{\text{XA}}\otimes(a\ket{0}_{\text{B}}-b\ket{1}_{\text{B}})\right.\notag\\
  &+\left.\ket{\psi^{+}}_{\text{XA}}\otimes(a\ket{1}_{\text{B}}+b\ket{0}_{\text{B}}) + \ket{\psi^{-}}_{\text{XA}}\otimes(a\ket{1}_{\text{B}}-b\ket{0}_{\text{B}})\right]
\end{align}
となる。
送信者はXとAの2つの量子ビットに対し、4つのBell基底$\qty{\ket{\phi^+},\,\ket{\phi^{-}},\,\ket{\psi^{+}},\,\ket{\psi^{-}}}$のどの状態なのかを決める測定(\textbf{Bell状態測定}、Bell state measurement)
を行ったとする。送信者の測定結果$\qty{\ket{\phi^+},\,\ket{\phi^{-}},\,\ket{\psi^{+}},\,\ket{\psi^{-}}}$に応じ、受信者が$\qty{I,\,\sigma_z,\,\sigma_x,\,\sigma_y}$のユニタリ変換を施すことで、
量子ビットXの状態をBの量子ビット上に転送することができる。

ここで、送信者の操作も受信者の操作も状態$\ket{\psi}$に依存していない。これは、転送する状態$\ket{\psi}$が何であるかをまったく知ることなく転送が行えることを意味している。
一般に$\ket{\psi}$の状態が1つしかないとすると、そこから測定によって$a$と$b$の値を得ることはできない。(測定結果の確率分布になるため。)
すなわち、未知の量子状態を古典通信だけで転送することは不可能である。量子テレポーテーションは、エンタングル状態を利用することで転送を可能にしている。
また、量子ビットをそのまま遠隔地に伝送しようとすると、伝送途中でデコヒーレンスの影響を受け状態は容易に壊れてしまう。
このように壊れやすい量子状態を、予め共有しておいたエンタングル状態と古典通信で遠隔地へ転送できることも、量子テレポーテーションの大きな利点であり、
量子暗号通信の長距離伝送(\textbf{量子中継}、quantum repeater)も可能にする。

量子テレポーテーションは光より速い速度で状態転送が行えるわけではないことにも注意する。
もし、受信者が送信者の測定結果を知らないとすると、量子ビットBの状態は$\ket{\phi^{+}}_{\text{AB}}$の縮約密度演算子$\sigma_{B} = I/2$のままであり、
受信者には何の情報も伝わらない。送信者の測定結果が古典通信で伝えられて初めて、受信者は状態$\ket{\psi}$を手にすることができる。

また、最大エンタングル状態における局所変換の性質を用いると、量子テレポーテーションは混合状態$\sigma$やエンタングル状態$\ket{\psi}\in\mathcal{H}_{\text{X}}\otimes\mathcal{H}_{\text{Y}}$も
忠実に転送できる。
\footnote{
4つのBell基底$\ket{\Psi_i}$は、$\ket{\Psi} = \ket{\phi^+}$として、ユニタリ行列$U_i$を用いて$\ket{\Psi_i} = (U_i\otimes I)\ket{\Psi}$と表される。
送信したい混合状態を$\sigma$として、その状態がどのヒルベルト空間にあるかを添字$\text{S}$で$\sigma_{\text{S}}$のように区別する。
AとXに対するBell状態測定後の状態が$\ket{\Psi_i}$である場合、受信者Bが受け取る状態は
\begin{equation}
  \sigma_\text{B}' = \frac{1}{4}U^{\dagger}_i \sigma_\text{B}U_i
\end{equation}
となるため、Bell状態測定結果を受け取り、適切なユニタリ演算子$U_i$を作用させると$\sigma_{\text{B}} = \sigma$を受け取ることができる。エンタングルした状態$\sigma_{\text{YX}}$の転送も同様に、
\begin{equation}
  \sigma_{\text{YB}}' = \frac{1}{4}(I_{\text{Y}}\otimes U^{\dagger}_i) \sigma_{\text{YB}}(I_{\text{Y}}\otimes U_i)
\end{equation}
と行われる。
}
\newpage

\subsubsection{超稠密符号}

\subsection{エンタングルメントの定量化}

\subsubsection{局所操作と古典通信(LOCC)}

\subsubsection{エンタングルメントの基本単位}

\subsubsection{エンタングルメントの濃縮}

\subsubsection{量子データ圧縮}

\subsubsection{エンタングルメントの希釈}

\subsubsection{純粋状態のエンタングルメント量}


\newpage
\subsection{Appendix}
\subsubsection{Schmidtの分解定理}
\vskip\baselineskip
\begin{tcolorbox}[
colback = white,
colframe = green!35!black,
fonttitle = \bfseries]
\begin{theorem}[Schmidtの分解定理]
  任意の合成状態$\ket{\psi}\in\mathcal{H}_1\otimes\mathcal{H}_2$に対して、
  $p_i>0\;(i=1,\ldots,\,l\leq\min[d_1,\,d_2])$と$\mathcal{H}_1,\,\mathcal{H}_2$の正規直交系$\qty{\ket{\zeta_i}}_{i=1}^{l}$と$\qty{\ket{\xi_j}}_{j=1}^{l}$が存在して
  \begin{equation}
    \ket{\psi} = \sum_{i=1}^{l}\sqrt{p_i}\ket{\zeta_i}\ket{\xi_i}
  \end{equation}
  と書くことができる。これをSchmidt分解(Schmidt decomposition)という。$p_i$をSchmidt係数(Schmidt coefficient)、$l$をSchmidt階数(Schmidt rank)と呼び、
  \begin{equation}
    \sum_{i=1}^{l}p_i = 1
  \end{equation}
  を満たす。
\end{theorem}
\end{tcolorbox}
\vskip\baselineskip


\begin{proof}
  一般性を失わず$d\coloneqq d_1 \leq d_2$とする。$\qty{\ket{\phi_i}}_{i=1}^{d}$と$\qty{\ket{\xi_j}}_{j=1}^{d_2}$をそれぞれ
  $\mathcal{H}_1$と$\mathcal{H}_2$の正規直交基底とすると、任意の状態$\ket{\psi}\in\mathcal{H}_1\otimes\mathcal{H}_2$は、
  \begin{equation}
    \ket{\psi} = \sum_{i=1}^{d}\sum_{j=1}^{d_2}\alpha_{ij}\ket{\phi_i}\otimes\ket{\xi_j} 
    = \sum_{i=1}^{d}\ket{\phi_i}\otimes\qty(\sum_{j=1}^{d_2}\alpha_{ij}\ket{\xi_j}) = \sum_{i=1}^{d}\ket{\phi_i}\otimes\ket{\chi_i}
  \end{equation}
  とかける。ただし、$\ket{\chi_i}\coloneqq \sum_{j=1}^{d_2}\alpha_{ij}\ket{\xi_j}$とした。
  ここで、$(i,j)$成分が$X_{ij}\coloneqq\braket{\chi_i}{\chi_j}$で与えられる$d\cross d$複素行列$X$を考える。
  定義より$X$は正値行列なので、$X$の固有値はすべて非負である。$X$の固有値を大きい順に$p_1,\,\ldots,\,p_d\geq 0$と置き、$l\;(\leq d)$個の固有値が正であるとする。
  また、$X$を対角化する$d\cross d$ユニタリ行列を$U$として、
  \begin{equation}
    \ket{\eta_j'}\coloneqq \sum_{i=1}^{d}U_{ij}\ket{\chi_i},\quad \ket{\zeta_k} \coloneqq \sum_{i=1}^{d}U^*_{ik}\ket{\phi_i}\qquad(j,k=1,\ldots,\,d)
  \end{equation}
  と置く。\footnote{本に誤植あり}
  $U$のユニタリ性により、
  \begin{equation}
    \sum_{k=1}^d U^{*}_{ik}\ket{\eta_k'} = \sum_{k=1}^d U^{*}_{ik}\qty(\sum_{j=1}^{d}U_{jk}\ket{\chi_j}) = \sum_{j=1}^{d}\delta_{ij}\ket{\chi_j} = \ket{\chi_i}
  \end{equation}
  を得る。したがって、
  \begin{equation}
    \ket{\psi} = \sum_{i=1}^{d}\ket{\phi_i}\otimes\ket{\chi_i} = \sum_{i=1}^{d}\ket{\phi_i}\otimes\qty(\sum_{k=1}^{d}U^{*}_{ik}\ket{\eta_k'})
               = \sum_{i=1}^{d}\sum_{k=1}^{d}U^{*}_{ik}\ket{\phi_i}\otimes\ket{\eta_k'} = \sum_{k=1}^{d}\ket{\zeta_k}\otimes\ket{\eta_k'}
  \end{equation}
  が成立する。
  \footnote{本の証明では$\qty{\ket{\zeta_k}}_{k=1}^{d}$が$\mathcal{H}_1$の正規直交基底であることを用いて示している。
  すなわち、$I=\sum_{k}\ketbra{\zeta_k}$と、$\braket{\zeta_j}{\phi_i} = U_{ij}$より
  \begin{equation}
    \ket{\psi} = \sum_{i=1}^{d}\ket{\phi_i}\otimes\ket{\chi_i} = \sum_{i=1}^d\qty(\sum_{j=1}^{d}\ketbra{\zeta_j})\ket{\phi_i}\otimes\qty(\sum_{k=1}^{d}U^*_{ik}\ket{\eta_k'})
               = \sum_{j,k=1}^{d}\qty(\sum_{k=1}^d U_{ij}U^{*}_{ik})\ket{\zeta_j}\otimes\ket{\eta_k'} = \sum_{k=1}^{d}\ket{\zeta_k}\otimes\ket{\eta_k'}
  \end{equation}
  と示している。}
  \\
  ところで、$X$の対角化
  \begin{equation}
    \sum_{k,l}U^{*}_{ki}X_{kl}U_{lj} = p_i\delta_{ij}
  \end{equation}
  より、
  \begin{equation}
    \braket{\eta_i'}{\eta_j'} = p_i\delta_{ij}
  \end{equation}
  が成り立つため、
  \begin{equation}
    \ket{\eta_i}\coloneqq \frac{1}{\sqrt{p_i}}\ket{\eta_i'}\quad(i=1,\ldots,\,l)
  \end{equation}
  と置けば、$\qty{\ket{\eta_i}}_{i=1}^{l}$は$\mathcal{H}_2$の正規直交系(正規直交基底ではない)を成す。
  $\braket{\psi}=1$を用いると、$\sum_{i=1}^{l}p_i=1$も示される。以上より、Schmidtの分解定理が示された。

  
  
  
\end{proof}




\end{document}