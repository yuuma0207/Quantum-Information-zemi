\RequirePackage{plautopatch}
\documentclass[11pt,aspectratio=169,xcolor=dvipsnames,table,dvipdfmx]{beamer}
\usepackage{bxdpx-beamer} % dvipdfmxなので必要
\usepackage{appendixnumberbeamer}

%Beamerの設定
\usetheme{Boadilla}

%Beamerフォント設定
\usefonttheme{professionalfonts} % Be professional!
\usepackage[T1]{fontenc}
\usepackage{mlmodern}  % 太いComputer Modern
% MLmodernのバグを修正: cf. https://tex.stackexchange.com/questions/646333/size-of-integral-symbol-in-section-header-with-mlmodern
\DeclareFontFamily{OMX}{mlmex}{}
\DeclareFontShape{OMX}{mlmex}{m}{n}{%
   <->mlmex10%
   }{} 
\usepackage{newtxtext} % 数式以外をTXフォントで上書き
\usepackage[deluxe,uplatex]{otf} % 日本語多ウェイト化
\usepackage{physics,bm}
\usepackage{color}
\usepackage{mhchem}
\usepackage{amsmath,amsfonts,amssymb,mathtools,amsthm}
\DeclareMathOperator{\sgn}{sgn}
\renewcommand{\familydefault}{\sfdefault}  % 英文をサンセリフ体に
\renewcommand{\kanjifamilydefault}{\gtdefault}  % 日本語をゴシック体に
\usefonttheme{structurebold} % タイトル部を太字
\setbeamerfont{alerted text}{series=\bfseries} % Alertを太字
\setbeamerfont{section in toc}{series=\mdseries} % 目次は太字にしない
\setbeamerfont{frametitle}{size=\Large} % フレームタイトル文字サイズ
\setbeamerfont{title}{size=\LARGE} % タイトル文字サイズ
\setbeamerfont{date}{size=\small}  % 日付文字サイズ

% Babel (日本語の場合のみ・英語の場合は不要)
\uselanguage{japanese}
\languagepath{japanese}
\deftranslation[to=japanese]{Theorem}{定理}
\deftranslation[to=japanese]{Lemma}{補題}
\deftranslation[to=japanese]{Example}{例}
\deftranslation[to=japanese]{Examples}{例}
\deftranslation[to=japanese]{Definition}{定義}
\deftranslation[to=japanese]{Definitions}{定義}
\deftranslation[to=japanese]{Problem}{問題}
\deftranslation[to=japanese]{Solution}{解}
\deftranslation[to=japanese]{Fact}{事実}
\deftranslation[to=japanese]{Proof}{証明}
\def\proofname{証明}

%Beamer色設定
\definecolor{UniBlue}{RGB}{0,150,200} 
\definecolor{AlertOrange}{RGB}{255,76,0}
\definecolor{AlmostBlack}{RGB}{38,38,38}
\setbeamercolor{normal text}{fg=AlmostBlack}  % 本文カラー
\setbeamercolor{structure}{fg=UniBlue} % 見出しカラー
\setbeamercolor{block title}{fg=UniBlue!50!black} % ブロック部分タイトルカラー
\setbeamercolor{alerted text}{fg=AlertOrange} % \alert 文字カラー
\mode<beamer>{
    \definecolor{BackGroundGray}{RGB}{254,254,254}
    \setbeamercolor{background canvas}{bg=BackGroundGray} % スライドモードのみ背景をわずかにグレーにする
}

%フラットデザイン化
\setbeamertemplate{blocks}[rounded] % Blockの影を消す
\useinnertheme{circles} % 箇条書きをシンプルに
\setbeamertemplate{navigation symbols}{} % ナビゲーションシンボルを消す
\setbeamertemplate{footline}[frame number] % フッターはスライド番号のみ
\setbeamerfont{footline}{size=\small}    % フッターを大きくした

%タイトルページ
\setbeamertemplate{title page}{%
    \vspace{2.5em}
    {\usebeamerfont{title} \usebeamercolor[fg]{title} \inserttitle \par}
    {\usebeamerfont{subtitle}\usebeamercolor[fg]{subtitle}\insertsubtitle \par}
    \vspace{1.5em}
    \begin{flushright}
        \usebeamerfont{author}\insertauthor\par
        \usebeamerfont{institute}\insertinstitute \par
        \vspace{3em}
        \usebeamerfont{date}\insertdate\par
      \end{flushright}
      \vspace{-6em}
      \begin{flushleft}
        \usebeamercolor[fg]{titlegraphic}\inserttitlegraphic
      \end{flushleft}
}

% Algorithm系
\usepackage{algorithm}
\usepackage[noend]{algorithmic}
\algsetup{linenosize=\color{fg!50}\footnotesize}
\renewcommand\algorithmicdo{:}
\renewcommand\algorithmicthen{:}
\renewcommand\algorithmicrequire{\textbf{Input:}}
\renewcommand\algorithmicensure{\textbf{Output:}}

% 定理
\theoremstyle{definition}
\newenvironment{mythm}{\begin{alertblock}{定理}}{\end{alertblock}} %自分の結果は赤色で表示

\AtBeginSection[]{
    \frame{\tableofcontents[currentsection, hideallsubsections]} %目次スライド
}
\usepackage[absolute,overlay]{textpos}
\usepackage{array} % needed for \arraybackslash
\usepackage{adjustbox} % for \adjincludegraphics

\usepackage{tikz}
\usetikzlibrary{calc}
\makeatletter
\chardef\zenkakuSpace=\jis"2121\relax % 和文ゴースト用全角スペース

\def\serifRoman#1{%
  \ifnum0\ifnum#1<1 1\fi\ifnum#1>12 1\fi>0 %%% 引数が 1~12 の範囲外ならエラー
    \@latex@error{\string\serifRoman: #1 must be 1..12}\@ehc
  \fi
  \zenkakuSpace\kern-1zw\relax %%% ゴースト処理(入口側)
  \begingroup
  \@tempcnta="215F\relax
  \advance\@tempcnta by #1\relax
  \kchardef\@roman@char=\ucs\@tempcnta\relax %% 今回対象のローマ数字
  \kchardef\@roman@I=\ucs"2160\relax %% ヒゲとして利用するローマ数字Ⅰ
  \@tempdima=-.132zw\relax
  \@tempdimb=.134zw\relax
  \@tempdimc=1zw\relax
  \setbox1\hbox to 1zw{\yoko\hfill
    \begin{tikzpicture}[baseline=(A.base),inner sep=0pt, outer sep=0pt]%
    \path[use as bounding box] (-.5\@tempdimc, -.5\@tempdimc) rectangle (.5\@tempdimc, .5\@tempdimc);
    \node (A) {\@roman@char};
    \ifcase#1
      \or % Ⅰ
        \@serifRoman@a{.45}{0}% 上ヒゲ
        \@serifRoman@b{.45}{0}% 下ヒゲ
      \or % Ⅱ
        \@serifRoman@a{.80}{0}% 上ヒゲ
        \@serifRoman@b{.80}{0}% 下ヒゲ
      \or % Ⅲ
        \@serifRoman@a{1.10}{0}% 上ヒゲ
        \@serifRoman@b{1.10}{0}% 下ヒゲ
      \or % Ⅳ
        \@serifRoman@a{.60}{-.24}% 左上ヒゲ
        \@serifRoman@a{.30}{.38}%% 右上ヒゲ
        \@serifRoman@b{.30}{-.35}% 左下ヒゲ
      \or % Ⅴ
        \@serifRoman@a{.30}{-.28}% 左上ヒゲ
        \@serifRoman@a{.30}{.28}%% 右上ヒゲ
      \or % Ⅵ
        \@serifRoman@a{.30}{-.38}% 左上ヒゲ
        \@serifRoman@a{.60}{.24}%% 右上ヒゲ
        \@serifRoman@b{.30}{.34}%% 右下ヒゲ
      \or % Ⅶ
        \@serifRoman@a{.30}{-.41}% 左上ヒゲ
        \@serifRoman@a{.75}{.20}%% 右上ヒゲ
        \@serifRoman@b{.52}{.28}%% 右下ヒゲ
      \or % Ⅷ
        \@serifRoman@a{.30}{-.41}% 左上ヒゲ
        \@serifRoman@a{.82}{.18}%% 右上ヒゲ
        \@serifRoman@b{.65}{.245}% 右下ヒゲ
      \or % Ⅸ
        \@serifRoman@a{.25}{.33}%% 右上ヒゲ
        \@serifRoman@a{.55}{-.22}% 左上ヒゲ
        \@serifRoman@b{.55}{-.22}% 左下ヒゲ
        \@serifRoman@b{.25}{.36}%% 右下ヒゲ
      \or % Ⅹ
        \@serifRoman@a{.25}{.23}%% 右上ヒゲ
        \@serifRoman@a{.25}{-.23}% 左上ヒゲ
        \@serifRoman@b{.25}{-.26}% 左下ヒゲ
        \@serifRoman@b{.25}{.26}%% 右下ヒゲ
      \or % Ⅺ
        \@serifRoman@a{.55}{.23}%% 右上ヒゲ
        \@serifRoman@a{.25}{-.34}% 左上ヒゲ
        \@serifRoman@b{.25}{-.36}% 左下ヒゲ
        \@serifRoman@b{.55}{.23}%% 右下ヒゲ
      \or % Ⅻ
        \@serifRoman@a{.75}{.20}%% 右上ヒゲ
        \@serifRoman@a{.25}{-.38}% 左上ヒゲ
        \@serifRoman@b{.25}{-.40}% 左下ヒゲ
        \@serifRoman@b{.75}{.20}%% 右下ヒゲ
    \fi
    \end{tikzpicture}%
    \hfill}%
  \box1\relax
  \endgroup
  \kern-1zw\relax\zenkakuSpace %%% ゴースト処理(出口側)
}
\def\@serifRoman@a#1#2{%
  \node[rotate=90,xscale=.7,yscale=#1] at ($(A.north) + (#2\@tempdimc,\@tempdima)$) {\@roman@I};
}
\def\@serifRoman@b#1#2{%
  \node[rotate=90,xscale=.7,yscale=#1] at ($(A.south) + (#2\@tempdimc,\@tempdimb)$) {\@roman@I};
}
\makeatother
%\usepackage{xcolor}

%タイトル
\title{基礎から学ぶ量子計算 アルゴリズムと計算量理論}
\subtitle{6.2 NPの量子版:QMA(後半:pp.194-pp.199)}
\author{\textbf{松本侑真}}
\date{2023/12/5}

\begin{document}
\maketitle

\begin{frame}{QMA問題の例}
  NP完全問題に対応するQMA完全問題\footnote{任意のQMA問題が多項式時間帰着可能であるようなQMA問題}
  としては、局所ハミルトニアンに関する次の問題\footnote{NP問題$k$-SATの量子版と考えられる}が代表的である:
  \begin{exampleblock}{問題6.3 (\textit{k}-LH (\textit{k}-local Hamiltonian))}
    \begin{itemize}
      \item \textbf{入力}:$n$量子ビット上の$k$局所ハミルトニアン
            \begin{equation}
              H=\sum_{j=1}^{r} H_j\qq{(ただし、$0\leq H_j \leq 1,\,r = \text{Poly}(n)$)}
            \end{equation}
            および、$\beta-\alpha \geq 1/{\text{poly}(n)}$
            を満たす実数$\alpha,\,\beta$
      \item \textbf{出力}:$H$の最小固有値が$\alpha$以下ならばYES、$\beta$以上ならばNO
    \end{itemize}
  \end{exampleblock}
\end{frame}

\begin{frame}
  \frametitle{補足:\textit{k}-LH問題はなぜ重要なのか?}
  結論からいうと、$k$-LH問題がQMA完全であることが、量子化学計算において量子加速
  \footnote{問題設定は現実的なものでなくても良いから、とりあえず最速の古典アルゴリズムよりも
  量子計算で高速に解けることを示すことが量子超越である。応用上意味のある問題での量子超越を量子加速という。}
  を得ることの難しさの本質だから。
  \begin{center}
    \adjincludegraphics[height=0.4\linewidth]{kLH_QMA.png}
  \end{center}
\end{frame}

\begin{frame}
  \frametitle{量子化学の目的の1つ:基底状態のエネルギーの計算}
  物理的なハミルトニアンは、局所ハミルトニアンの和で表される。
  遠隔相互作用を平均場とみなす近似や、多体相互作用を無視するため、
  ハミルトニアンの各項は定数個の電子(量子ビット)にしか作用しないためである。
  \begin{itemize}
    \item 局所磁場付きのFermi-Hubbard模型(強相関系の物理)
    \begin{equation}
      H_{\text{Hubb}}=-t\sum_{\langle i,j\rangle,\,s}a^{\dagger}_{i,s}a_{j,s} + U\sum_{i}n_{i,\uparrow}n_{i,\downarrow} 
        - \sum_{i}\bar{\sigma}_i\cdot\bm{B}_i
    \end{equation}
    \item 2次元イジング模型+横磁場(磁石のモデルとして基本的なもの)
    \begin{equation}
      H_{ZZXX} = \sum_{i}d_i\sigma^{x}_i + \sum_{i}h_i\sigma^{z}_i + \sum_{i,j}J_{i,j}\sigma^{z}_i\sigma^{z}_j + \sum_{i,j}K_{i,j}\sigma^{x}_i\sigma^{x}_j
    \end{equation}
  \end{itemize}
これらの基底状態におけるエネルギーが知りたい。
\begin{block}{決定問題とエネルギーの計算の関係}
  元々解きたかった問題:「与えられたハミルトニアンの基底所帯のエネルギーは{\color{red}\textbf{いくつ?}}」\\
  $k$-LH問題:「エネルギーは{\color{blue}\textbf{大きい or 小さい?}}」→~{\color{red}\textbf{二分探索でエネルギーを求められる。}}
\end{block}
\end{frame}

\begin{frame}
  \frametitle{化学計算において量子加速は得られないのか?}
$k$-LHはQMA完全であるため、一般には量子コンピュータでも効率的に(基底)エネルギーの決定問題は解けそうにない。

\begin{itemize}
  \item 2ローカルなハミルトニアンの方が物理的に自然である。物理的に自然なハミルトニアンに対応する$2$-LHを使えばいいのでは?
  \begin{itemize}
    \item 残念ながら、2体相互作用までのBoson/Fermion系や、並進/回転不変性があるハミルトニアンなど、自然な$2$-LHもQMA完全だとわかっている。
  \end{itemize}
  \item 密度汎関数法(DFT)中の(密度の汎関数としての)平均場の近似計算もBQP困難である。
\end{itemize}
  
\begin{exampleblock}{進展:基底状態の近似が与えられる場合の量子位相推定}
  ガイド付き$k$-LH問題(基底状態の近似が入力として与えられている$k$-LH問題)であれば、
  量子位相推定アルゴリズム(Long Termアルゴリズムの一つ)を用いて、基底エネルギー$E_{0,\,\text{est}}$を計算できる。
  $\abs{E_0 - E_{0,\,\text{est}}}<\epsilon$に対して、
  古典アルゴリズムでは$\epsilon$が定数の場合なら効率よく解けるが、
  量子では$\epsilon=1/\text{poly(n)}$の場合でも効率的に解ける。
  つまり、{\color{red}\textbf{精度が必要な計算において量子加速が期待される。}}
\end{exampleblock}



\end{frame}

\begin{frame}{問6.9}
  \textbf{問題}:問題6.3の出力は、「ある$\ket{\psi}$が存在して、$\ev{H}{\psi}\leq \alpha$ならばYES、どんな$\ket{\psi}$についても
  $\ev{H}{\psi}\geq \beta$ならばNO」(★)と書き換えられることを示せ。
  \begin{block}{問6.9の解答}
    $H$の$i$番目の固有値と固有ベクトルの組を$(E_i,\ket{\psi_i})$として、$\{\ket{\psi_i}\}$は完全系を成すとする。
    最小固有値を$E_0$とすると、任意の$\ket{\psi}$に対して
    \begin{equation}
      \ev{H}{\psi} \geq \ev{H}{\psi_0} = E_0
    \end{equation}
    が成立する。
    \begin{itemize}
      \item このとき、$E_0\leq \alpha$ならば$\ket{\psi_0}$に対してYESを返せば良い。
      \item また、$\forall\ket{\psi}\!\qty[\ev{H}{\psi}\geq \beta] \Leftrightarrow E_0\geq \beta$が成立する。
    \end{itemize}
    以上より、問題6.3の出力は(★)と書き換えられることが示された。
  \end{block}

\end{frame}

\begin{frame}{例6.1}
  \begin{exampleblock}{2-LHと2-SATの対応}
    2-SAT式
    \begin{equation*}
      F(x_1,\,x_2,\,x_3) = (\lnot x_1\lor x_2)\land(x_1\lor x_3)\land(\lnot x_2\lor \lnot x_3)\land(\lnot x_1\lor \lnot x_3)
    \end{equation*}
    は、次のハミルトニアンの集合に対応させれば自然に2-LHの入力となる:
    \begin{equation}
      \left\{ \,
      \begin{aligned}
         & H_1 = \ketbra{10}{10}_{12} \coloneqq \ketbra{10}\otimes I \qq{(テキストに誤植あり)}  \\
         & H_2 = \ketbra{00}{00}_{13} \coloneqq \ketbra{0} \otimes I\otimes \ketbra{0} \\
         & H_3 = \ketbra{11}{11}_{23} \coloneqq I \otimes\ketbra{11}                   \\
         & H_4 = \ketbra{11}{11}_{13} \coloneqq \ketbra{1}\otimes I \otimes \ketbra{1}
      \end{aligned}
      \right.
    \end{equation}
  \end{exampleblock}
\end{frame}

\begin{frame}{2-LHと2-SATの対応の意味(自分なりの理解)}
  局所ハミルトニアン$H$は$H=\sum_{i=1}^4 H_i$である。
  \begin{itemize}
    \item 例えば$x_1=1,\,x_2=0$の場合、$\lnot x_1 \lor x_2=0$となるため$F(1,0,x_3)=0$である。
    \item これに対応して、$\ket{\psi}=\ket{1}\otimes\ket{0}\otimes\ket{b}$では$\ev{H}{\psi}>0$となる。
  \end{itemize}
  \vfill
  $F(x_1,\,x_2,\,x_3) = f_1\land f_2\land f_3\land f_4$と見たときに、
  $f_i=0$に対応する状態$\ket{\psi_i}$に対して、$\ev{H_i}{\psi_i}>0$になるように$H_i$を設計する。\\
  例えば$\alpha=0,\,\beta=1$とすると、この2-SATは2-LHに対応する。
  \footnote{一般にこのような$H_i$が存在しなくても、$F=0$になる状態に対してハミルトニアンの期待値を大きくし、
    $\alpha,\,\beta$を適切に設定すれば$k$-LHに対応させられると思う。}
\end{frame}

\begin{frame}{問6.10(前半)}
  \textbf{問題}:3-SAT式
  \begin{equation}
    F_1(x_1,\,x_2,\,x_3,\,x_4) = (x_1\lor x_2\lor x_3)\land(x_2\lor \lnot x_3\lor x_4)\land(x_1\lor \lnot x_2\lor \lnot x_4)
  \end{equation}
  に対応する3-LHの入力を与えよ。
  \begin{block}{問6.10(前半)の解答}
    \begin{equation*}
      \left\{ \,
      \begin{aligned}
        f_1 & \coloneqq (x_1\lor x_2\lor x_3) = 0             &  & \Leftrightarrow (x_1,\,x_2,\,x_3,\,x_4) = (0,\,0,\,0,\,b) \\
        f_2 & \coloneqq (x_2\lor \lnot x_3\lor x_4) = 0       &  & \Leftrightarrow (x_1,\,x_2,\,x_3,\,x_4) = (b,\,0,\,1,\,0) \\
        f_3 & \coloneqq (x_1\lor \lnot x_2\lor \lnot x_4) = 0 &  & \Leftrightarrow (x_1,\,x_2,\,x_3,\,x_4) = (0,\,1,\,b,\,1)
      \end{aligned}
      \right.
    \end{equation*}
    である。$f_i =0$に対応する量子ビットを与えたときに、Hamiltonianの期待値が正になるように設計する。つまり、
    \begin{equation*}
      H_1 = \ketbra{000}\otimes I,\quad H_2 = I \otimes \ketbra{010},\quad H_3 = \ketbra{01}\otimes I\otimes \ketbra{1}
    \end{equation*}
    とすれば良い。
  \end{block}
\end{frame}

\begin{frame}{問6.10(後半)}
  \textbf{問題}:2-SAT式
  \begin{equation}
    F_2(x_1,\,x_2,\,x_3,\,x_4) = (x_1\lor \lnot x_3)\land(x_2\lor \lnot x_4)\land(\lnot x_1\lor \lnot x_2)
  \end{equation}
  に対応する2-LHの入力を与えよ。
  \begin{block}{問6.10(後半)の解答}
    \begin{equation*}
      \left\{ \,
      \begin{aligned}
        f_1 & \coloneqq (x_1\lor \lnot x_3)= 0        &  & \Leftrightarrow (x_1,\,x_2,\,x_3,\,x_4) = (0,\,b,\,1,\,b) \\
        f_2 & \coloneqq (x_2\lor \lnot x_4) = 0       &  & \Leftrightarrow (x_1,\,x_2,\,x_3,\,x_4) = (b,\,0,\,b,\,1) \\
        f_3 & \coloneqq (\lnot x_1\lor \lnot x_2) = 0 &  & \Leftrightarrow (x_1,\,x_2,\,x_3,\,x_4) = (1,\,1,\,b,\,b)
      \end{aligned}
      \right.
    \end{equation*}
    である。$f_i =0$に対応する量子ビットを与えたときに、Hamiltonianの期待値が正になるように設計する。つまり、
    \begin{equation*}
      H_1 = \ketbra{0}\otimes I\otimes \ketbra{1}\otimes I,\quad H_2 = I \otimes \ketbra{0}\otimes I\otimes \ketbra{1},\quad H_3 = \ketbra{1}\otimes \ketbra{1}\otimes I\otimes I
    \end{equation*}
    とすれば良い。
  \end{block}
\end{frame}

\begin{frame}{\textit{k}-LHがQMAに属することの保障}
  \begin{exampleblock}{\textit{k}-LHに対するQMAプロトコル}
    レジスタR上に証拠の候補が$\ket{\psi}_{\text{R}}$と与えられたとする。
    \begin{enumerate}
      \item レジスタAに一様重ね合わせ状態
            \begin{equation*}
              \frac{1}{\sqrt{r}}\sum_{j=1}^r \ket{j}_{\text{A}}
            \end{equation*}
            を準備する。
      \item レジスタAの値が$j$のとき、レジスタR上でPOVM$\{H_j,\,I-H_j\}$を実行する。
            そして、POVMの要素$H_j$に対応する測定値を得たときにreject、$I-H_j$に対応する測定値を得たときにaccept。
    \end{enumerate}
  \end{exampleblock}
\end{frame}

\begin{frame}{\textit{k}-LHがQMAに属することの保障}
  検証者がrejectを出力する確率は、レジスタR上の状態を$\ket{\psi_R}$として
  \begin{equation*}
    \Pr\qty[\text{reject}](\ket{\psi_R}) = \frac{1}{r}\sum_{j=1}^{r}\ev{H_j}{\psi_R} = \frac{1}{r}\ev{H}{\psi_R}
  \end{equation*}
  となる。YESの場合は、ある状態$\ket{\varphi}$が存在して
  \begin{equation*}
    \Pr\qty[\text{reject}](\ket{\varphi}) = \frac{1}{r}\ev{H}{\varphi} \leq \frac{\alpha}{r} \quad\qty(\Pr\qty[\text{accept}](\ket{\varphi})\geq 1-\frac{\alpha}{r})
  \end{equation*}
  となる。NOの場合は、全ての状態$\ket{\psi}$に対して
  \begin{equation*}
    \Pr\qty[\text{reject}](\ket{\psi}) = \frac{1}{r}\ev{H}{\psi} \geq \frac{\beta}{r} \quad\qty(\Pr\qty[\text{accept}](\ket{\psi})\leq 1-\frac{\beta}{r})
  \end{equation*}
  となる。($k$-LHの等価な言い換え(問6.9)を用いた。)
\end{frame}
\begin{frame}{\textit{k}-LHがQMAに属することの保障(続き)}
  まとめると、$x\in A_{\text{yes}},\,x\in A_{\text{no}}$それぞれに対応して
  \begin{equation}
    \exists{\ket{\varphi}}\qty[\Pr\qty[\text{accept}](\ket{\varphi}) \geq 1- \frac{\alpha}{r} = a] ,\quad \forall \ket{\psi}\qty[\Pr\qty[\text{accept}](\ket{\psi}) \leq 1- \frac{\beta}{r} = b]
  \end{equation}
  となるため、$k$-LHはQMA(a,b)に属する。また、$\beta-\alpha\geq 1/\text{poly}(n)$であるため、
  \begin{equation}
    a-b = \frac{\beta-\alpha}{r} \geq \frac{1}{\text{poly}(n)}
  \end{equation}
  となる。$a,\,b = 1-1/\text{poly}(n)$であることと、
  \begin{equation}
    \frac{1}{\text{exp}(n)} \leq 1-\frac{1}{\text{poly}(n)} \leq 1-\frac{1}{\text{exp}(n)}
  \end{equation}
  が成立することより命題6.11が適用でき、$k$-LHはQMAに属することが示された。
\end{frame}

\begin{frame}{\textit{k}-LHがQMA困難であることについて}
  $k$-LHがQMA困難である
  \footnote{$A$がQMA困難とは、任意の$B\in \text{QMA}$に対して、$B\leq_{p} A$が成立すること}ことは、
  Cook-Levin定理\footnote{SATはNP完全であるという定理}の証明を量子化すれば示せる。
  \begin{block}{証明の方針}
    \begin{enumerate}
      \item 任意の$A=(A_{\text{yes}},\,A_{\text{no}})\in \text{QMA}$を$k$-LHに帰着させるため、$A$を検証する一様回路族$\{Q_x\}$を考える。
      \item 量子回路$Q_x$の$t$ステップ目の状態を$\ket{\psi_t} =\prod_{u=1}^t U_{x,u}\ket{\varphi}\ket{0}^m$とする。このときの履歴状態
            \begin{equation}
              \frac{1}{\sqrt{T+1}}\sum_{t=0}^{T}\ket{t}_{\text{A}}\ket{\psi_t}_{\text{B}}\qq{($T=T(\abs{x})$は$Q_x$のサイズ)}
            \end{equation}
            が最低エネルギー状態になるように、局所ハミルトニアン$H$の各項を構成する。
    \end{enumerate}
  \end{block}
\end{frame}

\begin{frame}{任意のQMAを\textit{k}-LHに帰着させる}
  \begin{itemize}
    \item  局所ハミルトニアン$H$には
          \begin{enumerate}
            \item 検証者のアンシラ(補助量子ビット)が正しく$\ket{0}^m$にセットされているか
            \item 各量子ゲートが正しく適用されているか
            \item 終状態$\ket{\psi_t}$を測定したときにacceptを出力するか
          \end{enumerate}
          をチェックする項を含んでいる。
    \item 最初の2つは正しく$Q_x$が実行できるかどうかに関わっているため、満たされていない場合のペナルティを大きくする。
    \item  その結果、$Q_x$の毎ステップ後の状態の一様重ね合わせである履歴状態が最もペナルティを食わないようになる。
    \item 3番目のチェックに対応する項のペナルティで、YES入力とNO入力の最低エネルギーの差が出る仕組みになる。
  \end{itemize}
\end{frame}

\begin{frame}{quantum MAX-CUT}
  \small2-LHから派生する次の問題がQMA完全(QMA困難かつQMA)であることが示されている:
  \begin{exampleblock}{問題6.4 (quantum MAX-CUT)}
    \begin{itemize}
      \item \textbf{入力}:グラフ$G=(V,E)$および各辺$\{u,\,v\}\in E$に対する重み$w_{uv}>0$
      \item \textbf{出力}:
            \begin{equation}
              H_{G}\coloneqq \sum_{\{u,\,v\}\in E} w_{uv}\frac{I-X_uX_v - Y_uY_v - Z_uZ_v}{4}
            \end{equation}
            の最大固有値
    \end{itemize}
  \end{exampleblock}
  \small{NP完全問題であるMAX-CUTは\footnote{3回目講義「量子アルゴリズムの基礎」を参照}
    \begin{equation}
      H_G = \sum_{\{u,\,v\}\in E} w_{uv}\frac{I-Z_uZ_v}{2}
    \end{equation}
    の最大固有値を見つける問題と表現できるため、これの自然な量子版と見れる。}
\end{frame}

\frame{\centering{\Huge{Thank you for your attention!}}}




\end{document}