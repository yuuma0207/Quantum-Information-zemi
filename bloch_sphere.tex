%ctrl+alt+m:open Math Preview
\documentclass[a4paper,11pt,uplatex]{jsarticle}%titlepage
%:/usr/local/texlive/texmf-local/tex/latex/report/report.sty
\usepackage{myreport}
\title{Bloch球と量子状態について}
\author{松本侑真}
\date{\today}
\begin{document}
\maketitle
\begin{abstract}
$1$量子ビットの状態をBloch球上の点として表現する方法について、基礎事項をまとめる。
\end{abstract}
\tableofcontents
\newpage

\section{量子状態について}
\begin{itemize}
  \item $n$量子ビットの状態は$(\mathbb{C}^2)^{\otimes n}$の元で表される。ゼロベクトル($\ket{0}\otimes\ket{0}\cdots\otimes{\ket{0}}$ではない)は、
  どのような状態も表さない。
  \item $\alpha\in\mathbb{C}\setminus\{\bm{0}\}$とする。このとき、
  \begin{equation}
    (\mathbb{C}^{2})^{\otimes n}\setminus\{\bm{0}\}\ni\ket{x}\sim\alpha\ket{x}\in(\mathbb{C}^{2})^{\otimes n}\setminus\{\bm{0}\}
  \end{equation}
  の同値類は同一状態を表す。したがって、量子状態$\ket{\psi}$のノルムは1と約束する:
  \begin{equation}
    \ket{\psi}\in(\mathbb{C}^2)^{\otimes{n}}\setminus\{\bm{0}\}\wedge \braket{\psi}=1\;。
  \end{equation}
  \item $\ket{x},\,\ket{y}$を量子状態とする。このとき、$\alpha,\,\beta\in\mathbb{C}$として、
  \begin{equation}
    \alpha\ket{x} + \beta\ket{y}\quad(\abs{\alpha}^2+\abs{\beta}^2\neq 0)
  \end{equation}
  は量子状態である。
\end{itemize}

\subsection{1量子ビット系}
$1$qubitの量子状態$\mathbb{C}^2\setminus\{\bm{0}\}$の元は、$\alpha,\,\beta\in\mathbb{C}$を用いて
\begin{equation}
  \alpha\ket{0}+\beta\ket{1}\quad(\abs{\alpha}^2+\abs{\beta}^2 = 1,\,\braket{i}{j}=\delta_{ij}\;\;(i,j=0,1))
\end{equation}
と表す。$a,b\geq 0,\,0\leq t-s<2\pi$を用いて
\begin{equation}
  \alpha=ae^{is},\,\beta=be^{it}
\end{equation}
とおくと、
\begin{equation}
  a=\cos\frac{\theta}{2},\,b=\sin\frac{\theta}{2}\quad(0\leq\theta\leq \pi)
\end{equation}
と表すことができる。したがって、
\begin{equation}
  \alpha\ket{0}+\beta\ket{1} = e^{is}\qty(\cos\frac{\theta}{2}\ket{0} + e^{i(t-s)}\sin\frac{\theta}{2}\ket{1}) 
  = \cos\frac{\theta}{2}\ket{0} + e^{i\varphi}\sin\frac{\theta}{2}\ket{1}\quad(0\leq \varphi < 2\pi)
\end{equation}
と、$\theta,\,\varphi$を用いて記述することができる。これをBloch球表現と呼ぶ。
$\theta,\,\varphi$をこのように設定することで、1qubitの量子状態とBloch球上の点を1対1に対応させることができる。

\newpage
\section{Bloch球表現}
$0\leq\theta\leq\pi,\,0\leq \varphi<2\pi$として、
\begin{equation}
  \cos\frac{\theta}{2}\ket{0}+e^{i\varphi}\sin\frac{\theta}{2}\ket{1}\longmapsto (\sin\theta\cos\varphi,\,\sin\theta\sin\varphi,\,\cos\theta)
\end{equation}
という写像は、1qubitの状態と$\mathbb{R}^3$の単位球面(Bloch球)との1対1対応を与える。
例えば、Bloch球の北極($\theta=0$)は$\ket{0}$の量子状態、南極($\theta=\pi$)は$\ket{1}$の量子状態に対応している。
この表現を用いることで、量子ゲート(ユニタリ演算子)の作用をBloch球上の回転操作と同一視することができる。

\subsection{立体射影}
1qubitの量子状態$\ket{\psi}$からBloch球表現への写像の具体形を導出する。
\begin{equation}
  \ket{\psi}=\cos\frac{\theta}{2}\ket{0}+e^{i\varphi}\sin\frac{\theta}{2}\ket{1}
\end{equation}
の$\ket{0}$と$\ket{1}$の係数比
\begin{eqnarray}
  \alpha=\frac{e^{i\varphi}\sin\frac{\theta}{2}}{\cos\frac{\theta}{2}} = e^{i\varphi}\tan\frac{\theta}{2}
\end{eqnarray}
は、$\ket{\psi}$と1対1に対応する。$0\leq\theta\leq\pi$より、$\alpha$は任意の複素数もしくは$\infty$となる($\alpha\in\mathbb{C}\cup\{\infty\}$)。
2つの複素数の比の集合($\mathbb{C}\times\mathbb{C}$を「比が等しい」という同値関係で割った集合で、$\mathbb{C}\cup\{\infty\}$と同一視できる)
を複素射影直線と呼び、$\mathbb{P}^1(\mathbb{C})$で表す。

$\mathbb{R}^3$における$xy$平面上の点$(x,y,0)$を複素数$x+iy$と見て、$xy$平面を複素平面と同一視する。
また、原点を中心とする単位球面を$S^2$とおく:
\begin{equation}
  S^2=\{(x,y,z)\in\mathbb{R}^3\mid x^2+y^2+z^2=1\}\;。
\end{equation}
$S^2$上の点N$(0,0,1)$を北極、S$(0,0,-1)$を南極と呼ぶ。

$\alpha\in\mathbb{C}$に対して、$\alpha$に対応する$\mathbb{R}^3$上の点をAとする。
直線SAと$S^2$の交点のうち、Sでない点をPとして、写像
\begin{equation}
  F\colon \mathbb{C}\ni\alpha\longmapsto \text{P}\in S^2\setminus\{\text{S}\}
\end{equation}
を定義する。この$F$を立体射影と呼ぶ。さらに、$F(\infty)=\text{S}$と定めると、写像
\begin{equation}
  F\colon \mathbb{C}\cup\{\infty\} = \mathbb{P}^{1}(\mathbb{C})\longrightarrow S^2
\end{equation}
は全単射となる。$F$の像としての$S^2$をRiemann球面と呼ぶ。

\subsection{状態ベクトルと立体射影}

\section{Bloch球上の回転操作}

\section{Hilbert-Schmidt表現}

\subsection{量子ゲート操作の表現行列}


\end{document}