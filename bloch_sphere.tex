%ctrl+alt+m:open Math Preview
\documentclass[a4paper,11pt,uplatex]{jsarticle}%titlepage
%:/usr/local/texlive/texmf-local/tex/latex/report/report.sty
\usepackage{myreport}
\title{Bloch球と量子状態について}
\author{松本侑真}
\date{\today}
\begin{document}
\maketitle
\begin{abstract}
$1$量子ビットの状態をBloch球上の点として表現する方法について、基礎事項をまとめる。
\end{abstract}
\tableofcontents
\newpage

\section{量子状態について}
\begin{itemize}
  \item $n$量子ビットの状態は$(\mathbb{C}^2)^{\otimes n}$の元で表される。ゼロベクトル($\ket{0}\otimes\ket{0}\cdots\otimes{\ket{0}}$ではない)は、
  どのような状態も表さない。
  \item $\alpha\in\mathbb{C}\setminus\{\bm{0}\}$とする。このとき、
  \begin{equation}
    (\mathbb{C}^{2})^{\otimes n}\setminus\{\bm{0}\}\ni\ket{x}\sim\alpha\ket{x}\in(\mathbb{C}^{2})^{\otimes n}\setminus\{\bm{0}\}
  \end{equation}
  の同値類は同一状態を表す。したがって、量子状態$\ket{\psi}$のノルムは1と約束する:
  \begin{equation}
    \ket{\psi}\in(\mathbb{C}^2)^{\otimes{n}}\setminus\{\bm{0}\}\wedge \braket{\psi}=1\;。
  \end{equation}
  \item $\ket{x},\,\ket{y}$を量子状態とする。このとき、$\alpha,\,\beta\in\mathbb{C}$として、
  \begin{equation}
    \alpha\ket{x} + \beta\ket{y}\quad(\abs{\alpha}^2+\abs{\beta}^2\neq 0)
  \end{equation}
  は量子状態である。
\end{itemize}

\subsection{1量子ビット系}
$1$qubitの量子状態$\mathbb{C}^2\setminus\{\bm{0}\}$の元は、$\alpha,\,\beta\in\mathbb{C}$を用いて
\begin{equation}
  \alpha\ket{0}+\beta\ket{1}\quad(\abs{\alpha}^2+\abs{\beta}^2 = 1,\,\braket{i}{j}=\delta_{ij}\;\;(i,j=0,1))
\end{equation}
と表す。$a,b\geq 0,\,0\leq t-s<2\pi$を用いて
\begin{equation}
  \alpha=ae^{is},\,\beta=be^{it}
\end{equation}
とおくと、
\begin{equation}
  a=\cos\frac{\theta}{2},\,b=\sin\frac{\theta}{2}\quad(0\leq\theta\leq \pi)
\end{equation}
と表すことができる。したがって、
\begin{equation}
  \alpha\ket{0}+\beta\ket{1} = e^{is}\qty(\cos\frac{\theta}{2}\ket{0} + e^{i(t-s)}\sin\frac{\theta}{2}\ket{1}) 
  = \cos\frac{\theta}{2}\ket{0} + e^{i\varphi}\sin\frac{\theta}{2}\ket{1}\quad(0\leq \varphi < 2\pi)
\end{equation}
と、$\theta,\,\varphi$を用いて記述することができる。これをBloch球表現と呼ぶ。
$\theta,\,\varphi$をこのように設定することで、1qubitの量子状態とBloch球上の点を1対1に対応させることができる。

\newpage
\section{Bloch球表現}
$0\leq\theta\leq\pi,\,0\leq \varphi<2\pi$として、
\begin{equation}
  \cos\frac{\theta}{2}\ket{0}+e^{i\varphi}\sin\frac{\theta}{2}\ket{1}\longmapsto (\sin\theta\cos\varphi,\,\sin\theta\sin\varphi,\,\cos\theta)
\end{equation}
という写像は、1qubitの状態と$\mathbb{R}^3$の単位球面(Bloch球)との1対1対応を与える。
例えば、Bloch球の北極($\theta=0$)は$\ket{0}$の量子状態、南極($\theta=\pi$)は$\ket{1}$の量子状態に対応している。
この表現を用いることで、量子ゲート(ユニタリ演算子)の作用をBloch球上の回転操作と同一視することができる。

\subsection{立体射影}
1qubitの量子状態$\ket{\psi}$からBloch球表現への写像の具体形を導出する。
\begin{equation}
  \ket{\psi}=\cos\frac{\theta}{2}\ket{0}+e^{i\varphi}\sin\frac{\theta}{2}\ket{1}
\end{equation}
の$\ket{0}$と$\ket{1}$の係数比
\begin{eqnarray}
  \alpha=\frac{e^{i\varphi}\sin\frac{\theta}{2}}{\cos\frac{\theta}{2}} = e^{i\varphi}\tan\frac{\theta}{2}
\end{eqnarray}
は、$\ket{\psi}$と1対1に対応する。$0\leq\theta\leq\pi$より、$\alpha$は任意の複素数もしくは$\infty$となる($\alpha\in\mathbb{C}\cup\{\infty\}$)。
2つの複素数の比の集合($\mathbb{C}\times\mathbb{C}$を「比が等しい」という同値関係で割った集合で、$\mathbb{C}\cup\{\infty\}$と同一視できる)
を複素射影直線と呼び、$\mathbb{P}^1(\mathbb{C})$で表す。

$\mathbb{R}^3$における$xy$平面上の点$(x,y,0)$を複素数$x+iy$と見て、$xy$平面を複素平面と同一視する。
また、原点を中心とする単位球面を$S^2$とおく:
\begin{equation}
  S^2=\{(x,y,z)\in\mathbb{R}^3\mid x^2+y^2+z^2=1\}\;。
\end{equation}
$S^2$上の点N$(0,0,1)$を北極、S$(0,0,-1)$を南極と呼ぶ。

$\alpha\in\mathbb{C}$に対して、$\alpha$に対応する$\mathbb{R}^3$上の点をAとする。
直線SAと$S^2$の交点のうち、Sでない点をPとして、写像
\begin{equation}
  F\colon \mathbb{C}\ni\alpha\longmapsto \text{P}\in S^2\setminus\{\text{S}\}
\end{equation}
を定義する。この$F$を立体射影と呼ぶ。さらに、$F(\infty)=\text{S}$と定めると、写像
\begin{equation}
  F\colon \mathbb{C}\cup\{\infty\} = \mathbb{P}^{1}(\mathbb{C})\longrightarrow S^2
\end{equation}
は全単射となる。$F$の像としての$S^2$をRiemann球面と呼ぶ。

\newpage
\begin{tcolorbox}[
colback = white,
colframe = green!35!black,
fonttitle = \bfseries]
\begin{theorem}[立体射影の表現]
立体射影$F$は
\begin{equation}
  F(x+iy)=\qty(\frac{2x}{1+x^2+y^2},\,\frac{2y}{1+x^2+y^2},\,\frac{1-x^2-y^2}{1+x^2+y^2})
\end{equation}
で与えられる。
\end{theorem}
\end{tcolorbox}

\subsubsection*{証明}
$F(x+iy)=(X,\,Y,\,Z)$とおく。PはSとAの内分点または外分点となっているため、
$SP:AP=t:1-t$とおくと、$\text{P}\neq\text{S}$より$t\neq 0$である。
$\bm{SP} = (X,\,Y,\,Z+1),\,\bm{PA}=(x-X,\,y-Y,\,-Z)$なので、
\begin{equation}
  X=tx,\quad Y=ty,\quad Z=t-1
\end{equation}
が成立する。また、Pは$S^2$上の点なので、$t$について
\begin{gather}
  X^2+Y^2+Z^2 = t^2(x^2+y^2)+(t-1)^2 = 1 \notag \\
  \therefore\; t=\frac{2}{1+x^2+y^2}
\end{gather}
を得る。

\subsection{状態ベクトルと立体射影}
$\text{状態ベクトルの集合}\longleftrightarrow \mathbb{P}^1(\mathbb{C})\longleftrightarrow S^2$
がそれぞれ1対1に対応している。
\begin{equation}
  \ket{\psi} = \cos\frac{\theta}{2}\ket{0} + e^{i\varphi}\sin\frac{\theta}{2}\ket{1}
\end{equation}
に対応する$\mathbb{P}^1(\mathbb{C})$の元は
\begin{equation}
  e^{i\varphi}\tan\frac{\theta}{2} = \cos\varphi\tan\frac{\theta}{2} + i\sin\varphi\tan\frac{\theta}{2} = x + iy
\end{equation}
であった。$x^2+y^2+1 = \tan^2(\theta/2) + 1 = \cos^{-2}(\theta/2)$より、
\begin{equation}
  F\qty(e^{i\varphi}\tan\frac{\theta}{2}) 
  = \cos^2\frac{\theta}{2}\qty(2\cos\varphi\tan\frac{\theta}{2},\,2\sin\varphi\tan\frac{\theta}{2},\,1-\tan^2\frac{\theta}{2})
  = (\sin\theta\cos\varphi,\,\sin\theta\sin\varphi,\,\cos\theta)
\end{equation}
を得る。これは、Bloch球上の極座標表示と一致し、Bloch球は立体射影から構成されることがわかった。

\newpage
\section{Bloch球上の回転操作}
量子状態に作用させるユニタリ操作において、$\abs{\bm{n}}=1$なる$\bm{n}\in\mathbb{R}^3$を用いて
\begin{equation}
  R_n(\theta) = \exp(-i\frac{\theta}{2}\bm{n}\cdot\bm{\sigma}) = \cos\frac{\theta}{2}-i\sin\frac{\theta}{2}(\bm{n\cdot\bm{\sigma}})
\end{equation}
と表されるものを考える。これを量子状態に作用させると、Bloch球上では、$\bm{n}$方向の$\theta$回転を引き起こす。

\subsubsection*{例:$\bm{n}=(0,0,1)$のとき($z$軸回転)}
\begin{equation}
  R_z(\alpha)=\exp(-i\frac{\alpha}{2}\sigma_z)
\end{equation}
に対して、$\sigma_z\ket{0}=\ket{0},\quad\sigma_z\ket{1}=-\ket{1}$になることに注意すると、
\begin{equation}
  R_z(\alpha)\qty(\cos\frac{\theta}{2}\ket{0} + e^{i\varphi}\sin\frac{\theta}{2}\ket{1}) 
  = e^{-i\alpha/2}\qty(\cos\frac{\theta}{2}\ket{0} + e^{i(\varphi+\alpha)}\sin\frac{\theta}{2}\ket{1}) 
\end{equation}
となり、Bloch球上では$(\theta,\,\varphi)\rightarrow(\theta,\,\varphi+\alpha)$の変換が行われている。
これは、$z$軸方向に$\alpha$回転していることに他ならない。
他の例として、$\bm{n}=(1/\sqrt{2},0,1/\sqrt{2})$に対する$R_{n}(\pi)$はHadamardゲートに対応している。

\newpage
\section{Hilbert-Schmidt表現}
状態ベクトルからBlochへの対応関係は少しわかりにくい。
ここでは、密度行列をHilbert-Schmidt表現で表すと、自然とBloch球との対応が見て取れることを確認する。

状態ベクトル$\ket{\psi}$に対応して、密度行列$\rho=\ketbra{\psi}$を定義する。
\begin{equation}
  \ket{0}=
  \begin{pmatrix}
   1 \\ 0 
  \end{pmatrix},\quad
  \ket{1} = 
  \begin{pmatrix}
    0 \\ 1
  \end{pmatrix}
\end{equation}
とすると、$\rho$は$\mathbb{C}$上の$2\times 2$行列として表現できる。また、
$M_2(\mathbb{C}) = \text{span}\{\sigma_0=I,\,\sigma_1=X,\,\sigma_2=Y,\,\sigma_3=Z\}$
であることが知られている。$A\in M_2(\mathbb{C})$の$\sigma_\mu\,(\mu=0,1,2,3)$での展開係数$s_{\mu}$は、
\begin{equation}
  s_\mu = \frac{1}{2}\Tr(A\sigma_{\mu})
\end{equation}
と求めることができる。この$s_\mu$を用いて、$A\in M_2(\mathbb{C})$を
\begin{equation}
  A=
  \begin{pmatrix}
    s_0 \\ s_1 \\ s_2 \\ s_3
  \end{pmatrix}_{\text{HS}}
\end{equation}
と表すことを、Hilbert-Schmidt表現と呼ぶ。すなわち、それぞれの基底の表現は
\begin{equation}
  I=
  \begin{pmatrix}
    1 \\ 0 \\ 0 \\ 0
  \end{pmatrix}_{\text{HS}},\,
  X=
  \begin{pmatrix}
    0 \\ 1 \\ 0 \\ 0
  \end{pmatrix}_{\text{HS}},\,
  Y=
  \begin{pmatrix}
    0 \\ 0 \\ 1 \\ 0
  \end{pmatrix}_{\text{HS}},\,
  Z=
  \begin{pmatrix}
    0 \\ 0 \\ 0 \\ 1
  \end{pmatrix}_{\text{HS}}
\end{equation}
となる。
\subsubsection*{準備}
密度行列の表現を求めるために、$\ketbra{i}{j}\,(i,j=0,1)$を$\sigma_\mu$で展開しておく:
\begin{align}
  \ketbra{0} &= 
  \begin{pmatrix}
    1 \\ 0
  \end{pmatrix}
  \begin{pmatrix}
    1 & 0
  \end{pmatrix}
  =
  \begin{pmatrix}
    1 & 0 \\ 0 & 0
  \end{pmatrix}
  =\frac{I+Z}{2}\;, \\
  \ketbra{1} &= 
  \begin{pmatrix}
    0 \\ 1
  \end{pmatrix}
  \begin{pmatrix}
    0 & 1
  \end{pmatrix}
  =
  \begin{pmatrix}
    0 & 0 \\ 0 & 1
  \end{pmatrix}
  =\frac{I-Z}{2}\;, \\
  \ketbra{0}{1} &= 
  \begin{pmatrix}
    1 \\ 0
  \end{pmatrix}
  \begin{pmatrix}
    0 & 1
  \end{pmatrix}
  =
  \begin{pmatrix}
    0 & 1 \\ 0 & 0
  \end{pmatrix}
  =\frac{X+iY}{2}\;。
\end{align}
1qubitの量子状態
\begin{equation}
  \ket{\psi}=\cos\frac{\theta}{2}\ket{0} + e^{i\varphi}\sin\frac{\theta}{2}\ket{1}
\end{equation}
は、Bloch球上の極座標表示$(\theta,\,\varphi)$に対応していた。密度行列を求めてみると、
\begin{align}
  \rho = \ketbra{\psi} &= \cos^2\frac{\theta}{2}\ketbra{0} + e^{-i\varphi}\sin\frac{\theta}{2}\cos\frac{\theta}{2}\ketbra{0}{1}
   + e^{i\varphi}\sin\frac{\theta}{2}\cos\frac{\theta}{2}\ketbra{1}{0} + \sin^2\frac{\theta}{2}\ketbra{1} \notag \\
   &= \frac{1}{2}
   \begin{pmatrix}
    I & X & Y & Z
   \end{pmatrix}
   \begin{pmatrix}
    1 \\ \sin\theta\cos\varphi \\ \sin\theta\sin\varphi \\ \cos\theta
   \end{pmatrix}
\end{align}
と表現できる。密度行列のHilbert-Schmidt表現を用いると、Bloch球上の点の座標が自然と出てくることがわかる。

\newpage
\subsection{量子ゲート操作の表現行列}
量子ゲートを作用させた際の密度行列の変換則を調べることで、量子ゲートがBloch球上でどのような回転操作に対応しているかを直感的に理解できる。
ユニタリ演算子を$U$とする。$\ket{\psi}\mapsto U\ket{\psi}$に対応して、密度演算子は
\begin{equation}
  \rho = \ketbra{\psi}\mapsto U\ketbra{\psi}U^{\dagger} =U\rho U^\dagger
\end{equation}
と変換される。したがって、$f_U(\rho) = U\rho U^{\dagger}$は、密度行列の変換を表す写像となっている。
\begin{equation}
  f_U(\rho) = f_U\qty(\sum_{\mu}s_\mu\sigma_\mu) = \sum_{\mu}s_\mu f_U(\sigma_{\mu})
\end{equation}
が成立するため、パウリ行列の変換を調べれば良い。
\subsubsection*{例1:$X$ゲート操作のHS基底における表現}
パウリ行列の反交換関係を用いると簡単に求めることができる:
\begin{equation}
  f_X(I) = XIX^{\dagger} = I,\,f_X(X) = XXX^{\dagger} = X,\,f_X(Y) = XYX^{\dagger} = -Y,\,f_X(Z) = XZX^{\dagger} = -Z\;。
\end{equation}
これより、$f_X(\cdot)$の表現行列は
\begin{equation}
  \begin{pmatrix}
    1 & 0 & 0 & 0 \\ 0 & 1 & 0 & 0 \\
    0 & 0 & -1 & 0 \\ 0 & 0 & 0 & -1
  \end{pmatrix}_{\text{HS}}
\end{equation}
となる。すなわち、$\rho = \ketbra{\psi} = 2^{-1}(1,\,x,\,y,\,z)_{\text{HS}}^{\top}$は$X$ゲート操作により、
\begin{equation}
  f_X(\rho) = 
  \begin{pmatrix}
    1 & 0 & 0 & 0 \\ 0 & 1 & 0 & 0 \\
    0 & 0 & -1 & 0 \\ 0 & 0 & 0 & -1
  \end{pmatrix}
  \begin{pmatrix}
    1/2 \\ x/2 \\ y/2 \\ z/2
  \end{pmatrix}
  =\frac{1}{2}
  \begin{pmatrix}
    1 \\ x \\ -y \\ -z
  \end{pmatrix}
\end{equation}
となる。これは、$x$軸方向に$\pi$回転していることを表す。$Y,\,Z$ゲートも同様に、その軸方向に$\pi$回転させる作用を持つ。

\subsubsection*{例2:$H$ゲート操作のHS基底における表現}
次に、$H$ゲートについて考える。$H$ゲートをパウリ行列の線形結合で書き表すと先ほどと同じような計算で求まるが、
ここでは別の方法を紹介する。
それは、$a\in\{0,1\}$として
\begin{equation}
  X\ket{a} = \ket{\bar{a}},\,Y\ket{a} = (-1)^ai\ket{\bar{a}},\,Z\ket{a} = (-1)^a\ket{a}
\end{equation}
が成立することや、
\begin{equation}
  X\ket{+} = \ket{+},\,X\ket{-}=-\ket{-},\quad Y\ket{i} = \ket{i},\,Y\ket{i-}=-\ket{i-}
\end{equation}
が成立することを用いる。(パウリ行列の固有値は$\pm 1$という性質がある。)

$H$ゲートは、$H=(X+Z)/\sqrt{2}$と表されること(や、$(1,0,1)$方向に$\pi$回転すること)を用いると、
\begin{gather}
  H\ket{0} = \frac{\ket{0}+\ket{1}}{\sqrt{2}} = \ket{+},\quad H\ket{1} = \frac{\ket{0}-\ket{1}}{\sqrt{2}} = \ket{-} \\
  H\ket{i} = \ket{i-},\quad H\ket{i-} = \ket{i}
\end{gather}
が成立する。


\end{document}